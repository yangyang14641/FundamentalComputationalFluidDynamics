%2multibyte Version: 5.50.0.2960 CodePage: 936
\documentclass{article}%
\usepackage{amsmath}
\usepackage{amsfonts}
\usepackage{amssymb}
\usepackage{graphicx}%
\setcounter{MaxMatrixCols}{30}
%TCIDATA{OutputFilter=latex2.dll}
%TCIDATA{Version=5.50.0.2960}
%TCIDATA{Codepage=936}
%TCIDATA{CSTFile=40 LaTeX article.cst}
%TCIDATA{Created=Wednesday, November 25, 2015 21:51:12}
%TCIDATA{LastRevised=Friday, November 27, 2015 14:46:24}
%TCIDATA{<META NAME="GraphicsSave" CONTENT="32">}
%TCIDATA{<META NAME="SaveForMode" CONTENT="1">}
%TCIDATA{BibliographyScheme=Manual}
%TCIDATA{<META NAME="DocumentShell" CONTENT="Standard LaTeX\Blank - Standard LaTeX Article">}
%BeginMSIPreambleData
\providecommand{\U}[1]{\protect\rule{.1in}{.1in}}
%EndMSIPreambleData
\newtheorem{theorem}{Theorem}
\newtheorem{acknowledgement}[theorem]{Acknowledgement}
\newtheorem{algorithm}[theorem]{Algorithm}
\newtheorem{axiom}[theorem]{Axiom}
\newtheorem{case}[theorem]{Case}
\newtheorem{claim}[theorem]{Claim}
\newtheorem{conclusion}[theorem]{Conclusion}
\newtheorem{condition}[theorem]{Condition}
\newtheorem{conjecture}[theorem]{Conjecture}
\newtheorem{corollary}[theorem]{Corollary}
\newtheorem{criterion}[theorem]{Criterion}
\newtheorem{definition}[theorem]{Definition}
\newtheorem{example}[theorem]{Example}
\newtheorem{exercise}[theorem]{Exercise}
\newtheorem{lemma}[theorem]{Lemma}
\newtheorem{notation}[theorem]{Notation}
\newtheorem{problem}[theorem]{Problem}
\newtheorem{proposition}[theorem]{Proposition}
\newtheorem{remark}[theorem]{Remark}
\newtheorem{solution}[theorem]{Solution}
\newtheorem{summary}[theorem]{Summary}
\newenvironment{proof}[1][Proof]{\noindent\textbf{#1.} }{\ \rule{0.5em}{0.5em}}
\begin{document}

\begin{center}
\textbf{Solve ODE BVP problem by Weighted Residual method}
\end{center}

\begin{flushright}
\textbf{Yang Yang}

\textbf{2015/11/26}
\end{flushright}

Define a ODE BVP problem:%
\begin{align*}
L\left(  u\right)   &  =\frac{d^{2}u}{dx^{2}}+u+x=0,\quad\left(  0\leq
x\leq1\right) \\
u\left(  0\right)   &  =u\left(  1\right)  =0
\end{align*}


in fact the problem has a exact solution%
\[
u\left(  x\right)  =\frac{\sin\left(  x\right)  }{\sin\left(  1\right)  }-x
\]


Next we find a test function satisfied boundary condition:%
\[
u=x\left(  1-x\right)  \left(  a_{0}+a_{1}x+a_{2}x^{2}+a_{3}x^{3}+a_{4}%
x^{4}+...+a_{n}x^{n}\right)
\]


Some derive

Residual of equation%
\[
R=Lu-f
\]


and we want the integral of residual equal to zero:
\begin{align*}
\left\langle R,\omega_{i}\right\rangle  &  =\int_{\Omega}R\omega_{i}%
d\Omega=0\\
\int_{\Omega}\left(  Lu-f\right)  \omega_{i}d\Omega &  =0
\end{align*}


\textbf{1.Compute by hand}

test function:%
\[
u=x\left(  1-x\right)  \left(  a_{0}+a_{1}x\right)
\]


here $a_{0}$ and $a_{1}$ are constant which need to be worked out.

(1) Collocation method:

Points:%
\[
x_{1}=\frac{1}{3},\quad x_{2}=\frac{2}{3}%
\]%
\begin{align*}
\int_{\Omega}\left(  Lu-f\right)  \omega_{i}d\Omega &  =0\\
Weight  &  :\omega_{i}=\delta\left(  x-x_{i}\right)
\end{align*}%
\begin{align*}
u  &  =x\left(  1-x\right)  \left(  a_{0}+a_{1}x\right) \\
\left(  Lu-f\right)   &  =-4\,a_{1}\,x-2\,a_{0}+2\,\left(  1-x\right)
a_{1}+x\left(  1-x\right)  \left(  a_{1}\,x+a_{0}\right)  +x
\end{align*}%
\begin{align*}
&  \int_{\Omega}\left(  Lu-f\right)  \omega_{i}d\Omega\\
&  =\int_{0}^{1}\left[  -4\,a_{1}\,x-2\,a_{0}+2\,\left(  1-x\right)
a_{1}+x\left(  1-x\right)  \left(  a_{1}\,x+a_{0}\right)  +x\right]
\delta\left(  x-x_{i}\right)  dx
\end{align*}%
\[
\int_{\Omega}\left(  Lu-f\right)  \omega_{i}d\Omega=0
\]


So we obtain linear algebra equations:%
\begin{align*}
\left.  \left[  -4\,a_{1}\,x-2\,a_{0}+2\,\left(  1-x\right)  a_{1}+x\left(
1-x\right)  \left(  a_{1}\,x+a_{0}\right)  +x\right]  \right\vert _{x=x_{1}}
&  =0\\
\left.  \left[  -4\,a_{1}\,x-2\,a_{0}+2\,\left(  1-x\right)  a_{1}+x\left(
1-x\right)  \left(  a_{1}\,x+a_{0}\right)  +x\right]  \right\vert _{x=x_{2}}
&  =0
\end{align*}%
\begin{align*}
\frac{2}{27}a_{1}-\frac{16}{9}a_{0}+\frac{1}{3}  &  =0\\
-\frac{50}{27}a_{1}-\frac{16}{9}a_{0}+\frac{2}{3}  &  =0
\end{align*}


Solve these two equation%
\[
a_{0}=\frac{81}{416},\quad a_{1}=\frac{9}{52}%
\]


Approximation Solution:%
\begin{align*}
u  &  =x\left(  1-x\right)  \left(  a_{0}+a_{1}x\right) \\
u  &  =x\left(  1-x\right)  \left(  \frac{81}{416}+\frac{9}{52}x\right)
\end{align*}


(2) Sub-domain method:

Sub-domains:%
\begin{align*}
D_{1}  &  :0<x<\frac{1}{2}\\
D_{2}  &  :0<x<1
\end{align*}%
\begin{align*}
\int_{\Omega}\left(  Lu-f\right)  \omega_{i}d\Omega &  =0\\
Weight  &  :\omega_{i}=\left\{
\begin{array}
[c]{c}%
1,\quad x\in D_{i}\\
0,\quad x\notin D_{i}%
\end{array}
\right.
\end{align*}%
\begin{align*}
&  \int_{\Omega}\left(  Lu-f\right)  \omega_{i}d\Omega\\
&  =\int_{0}^{1}\left[  -4\,a_{1}\,x-2\,a_{0}+2\,\left(  1-x\right)
a_{1}+x\left(  1-x\right)  \left(  a_{1}\,x+a_{0}\right)  +x\right]
\omega_{i}dx
\end{align*}%
\[
\int_{\Omega}\left(  Lu-f\right)  \omega_{i}d\Omega=0
\]%
\begin{align*}
\int_{0}^{\frac{1}{2}}\left[  -4\,a_{1}\,x-2\,a_{0}+2\,\left(  1-x\right)
a_{1}+x\left(  1-x\right)  \left(  a_{1}\,x+a_{0}\right)  +x\right]
\omega_{i}dx  &  =0\\
\int_{0}^{1}\left[  -4\,a_{1}\,x-2\,a_{0}+2\,\left(  1-x\right)
a_{1}+x\left(  1-x\right)  \left(  a_{1}\,x+a_{0}\right)  +x\right]
\omega_{i}dx  &  =0
\end{align*}


Then we obtain two linear algebra equations:%
\begin{align*}
\frac{53}{192}a_{1}-\frac{11}{12}a_{0}+\frac{1}{8}  &  =0\\
-\frac{11}{12}a_{1}-\frac{11}{6}a_{0}+\frac{1}{2}  &  =0
\end{align*}


Solve these equations%
\begin{align*}
a_{0}  &  =\frac{97}{517}\\
a_{1}  &  =\frac{8}{47}%
\end{align*}


Approximation solution%
\begin{align*}
u  &  =x\left(  1-x\right)  \left(  a_{0}+a_{1}x\right) \\
u  &  =x\left(  1-x\right)  \left(  \frac{97}{517}+\frac{8}{47}x\right)
\end{align*}


(3) Moment method:

Weight function:%
\[
\omega_{i}=\left\vert \mathbf{r}\right\vert ^{i-1}%
\]


and in our case%

\begin{align*}
\omega_{1}\left(  x\right)   &  =1\\
\omega_{2}\left(  x\right)   &  =x
\end{align*}%
\[
\int_{\Omega}\left(  Lu-f\right)  \omega_{i}d\Omega=0
\]%
\begin{align*}
\int_{0}^{\frac{1}{2}}\left[  -4\,a_{1}\,x-2\,a_{0}+2\,\left(  1-x\right)
a_{1}+x\left(  1-x\right)  \left(  a_{1}\,x+a_{0}\right)  +x\right]  dx  &
=0\\
\int_{0}^{1}\left[  -4\,a_{1}\,x-2\,a_{0}+2\,\left(  1-x\right)
a_{1}+x\left(  1-x\right)  \left(  a_{1}\,x+a_{0}\right)  +x\right]  xdx  &
=0
\end{align*}


we obtain two linearalgebra equations%
\begin{align*}
-\frac{11}{12}a_{1}-\frac{11}{6}a_{0}+\frac{1}{2}  &  =0\\
-\frac{19}{20}a_{1}-\frac{11}{12}a_{0}+\frac{1}{3}  &  =0
\end{align*}


Solve these linearalgebra equations we obtain%
\[
a_{0}=\frac{122}{649},\quad a_{1}=\frac{10}{59}%
\]


So the approximation solution is%
\[
u\left(  x\right)  =x\left(  1-x\right)  \left(  \frac{122}{649}x+\frac
{10}{59}\right)
\]


(4) Least-Square method:

Target functional:%

\[
J=\int_{\Omega}R^{2}d\Omega
\]


minimum Target functional%
\begin{align*}
\frac{\partial J}{\partial a_{i}}  &  =\frac{\partial}{\partial a_{i}}%
\int_{\Omega}R^{2}d\Omega\\
&  =\int_{\Omega}\frac{\partial}{\partial a_{i}}R^{2}d\Omega\\
&  =\int_{\Omega}2R\frac{\partial R}{\partial a_{i}}d\Omega=0
\end{align*}


So we know weight function is%
\[
\omega_{i}=\frac{\partial R}{\partial a_{i}}%
\]%
\begin{align*}
\omega_{0}  &  =-2+x\left(  1-x\right) \\
\omega_{1}  &  =-6x+2+x^{2}\left(  1-x\right)
\end{align*}
%

\[
\int_{\Omega}\left(  Lu-f\right)  \omega_{i}d\Omega=0
\]%
\begin{align*}
\int_{0}^{\frac{1}{2}}\left[  -4\,a_{1}\,x-2\,a_{0}+2\,\left(  1-x\right)
a_{1}+x\left(  1-x\right)  \left(  a_{1}\,x+a_{0}\right)  +x\right]
\omega_{0}dx  &  =0\\
\int_{0}^{1}\left[  -4\,a_{1}\,x-2\,a_{0}+2\,\left(  1-x\right)
a_{1}+x\left(  1-x\right)  \left(  a_{1}\,x+a_{0}\right)  +x\right]
\omega_{1}dx  &  =0
\end{align*}


Then we obtain two linear algebra equations:%
\begin{align*}
\frac{101}{60}a_{1}+\frac{101}{30}a_{0}-\frac{11}{12}  &  =0\\
\frac{131}{35}a_{1}+\frac{101}{60}a_{0}-\frac{19}{20}  &  =0
\end{align*}


Solve these equations:%
\begin{align*}
a_{0}  &  =\frac{46161}{246137}\\
a_{1}  &  =\frac{413}{2437}%
\end{align*}


Approximation result%
\begin{align*}
u  &  =x\left(  1-x\right)  \left(  a_{0}+a_{1}x\right) \\
u  &  =x\left(  1-x\right)  \left(  \frac{46161}{246137}+\frac{413}%
{2437}x\right)
\end{align*}


(5) Galerkin method:%
\[
u=x\left(  1-x\right)  \left(  a_{0}+a_{1}x\right)
\]%
\[
u=a_{0}x\left(  1-x\right)  +a_{1}x^{2}\left(  1-x\right)
\]%
\begin{align*}
\varphi_{0}  &  =x\left(  1-x\right) \\
\varphi_{1}  &  =x^{2}\left(  1-x\right)
\end{align*}%
\begin{align*}
\int_{\Omega}\left(  Lu-f\right)  \omega_{i}d\Omega &  =0\\
Wieght\quad function  &  :\omega_{i}=\varphi_{i}%
\end{align*}%
\begin{align*}
\int_{0}^{\frac{1}{2}}\left[  -4\,a_{1}\,x-2\,a_{0}+2\,\left(  1-x\right)
a_{1}+x\left(  1-x\right)  \left(  a_{1}\,x+a_{0}\right)  +x\right]
\omega_{0}dx  &  =0\\
\int_{0}^{1}\left[  -4\,a_{1}\,x-2\,a_{0}+2\,\left(  1-x\right)
a_{1}+x\left(  1-x\right)  \left(  a_{1}\,x+a_{0}\right)  +x\right]
\omega_{1}dx  &  =0
\end{align*}


So we obtain two linear algebra equations
\begin{align*}
-\frac{3}{20}a_{1}-\frac{3}{10}a_{0}+\frac{1}{12}  &  =0\\
-\frac{13}{105}a_{1}-\frac{3}{20}a_{0}+\frac{1}{20}  &  =0
\end{align*}


solve these equations%
\begin{align*}
a_{0}  &  =\frac{71}{369}\\
a_{1}  &  =\frac{7}{41}%
\end{align*}%
\[
u=x\left(  1-x\right)  \left(  \frac{71}{369}x+\frac{7}{41}\right)
\]


Compare plot of these weighted residual method:

\begin{center}%
%TCIMACRO{\FRAME{ftbpF}{6.5916in}{4.1675in}{0pt}{}{}{multimethodcompare.eps}%
%{\special{ language "Scientific Word";  type "GRAPHIC";
%maintain-aspect-ratio TRUE;  display "USEDEF";  valid_file "F";
%width 6.5916in;  height 4.1675in;  depth 0pt;  original-width 9.5415in;
%original-height 6.0174in;  cropleft "0";  croptop "1";  cropright "1";
%cropbottom "0";  filename 'MultiMethodCompare.eps';file-properties "XNPEU";}}
%}%
%BeginExpansion
\begin{figure}[ptb]%
\centering
\includegraphics[
height=4.1675in,
width=6.5916in
]%
{MultiMethodCompare.eps}%
\end{figure}
%EndExpansion



\end{center}

\bigskip

----------------------------------------------------------------------------------------------------------------------------------------------------

\bigskip

\textbf{2.Galerkin method for any given }$n$

approximation polynomial%
\[
u=x\left(  1-x\right)  \left(  a_{0}+a_{1}x+a_{2}x^{2}+a_{3}x^{3}+a_{4}%
x^{4}+...+a_{n}x^{n}\right)
\]


and base functions%
\begin{align*}
\varphi_{0}\left(  x\right)   &  =x\left(  1-x\right) \\
\varphi_{1}\left(  x\right)   &  =x^{2}\left(  1-x\right) \\
\varphi_{2}\left(  x\right)   &  =x^{3}\left(  1-x\right) \\
&  ...\\
\varphi_{n}\left(  x\right)   &  =x^{n+1}\left(  1-x\right)
\end{align*}


and approximation polynomial%
\[
u\left(  x\right)  =%
%TCIMACRO{\dsum \limits_{i=0}^{n}}%
%BeginExpansion
{\displaystyle\sum\limits_{i=0}^{n}}
%EndExpansion
a_{i}\varphi_{i}\left(  x\right)
\]


basic function%
\[
\omega_{i}\left(  x\right)  =\varphi_{i}\left(  x\right)
\]


Weighted residual method%
\begin{align*}
\int_{\Omega}\left(  Lu-f\right)  \omega_{i}d\Omega &  =0\\
Wieght\quad function  &  :\omega_{i}=\varphi_{i}%
\end{align*}


Then we'll get linear Algebra equations:%
\begin{align*}
f_{1}\left(  a_{0},a_{1},...,a_{n}\right)   &  =0\\
f_{2}\left(  a_{0},a_{1},...,a_{n}\right)   &  =0\\
f_{3}\left(  a_{0},a_{1},...,a_{n}\right)   &  =0\\
&  ...\\
f_{n}\left(  a_{0},a_{1},...,a_{n}\right)   &  =0
\end{align*}


Solve these linear algebra equations, we obtain these coefficients in the
approxiamtion polynomial.%
\begin{align*}
&  a_{0},a_{1},...,a_{n}\\
u  &  =x\left(  1-x\right)  \left(  a_{0}+a_{1}x+a_{2}x^{2}+a_{3}x^{3}%
+a_{4}x^{4}+...+a_{n}x^{n}\right)
\end{align*}


We using Maple to realize this procedure.

\begin{center}%
%TCIMACRO{\FRAME{ftbpF}{6.0701in}{4.1667in}{0pt}{}{}{galerkin.eps}%
%{\special{ language "Scientific Word";  type "GRAPHIC";
%maintain-aspect-ratio TRUE;  display "USEDEF";  valid_file "F";
%width 6.0701in;  height 4.1667in;  depth 0pt;  original-width 9.8191in;
%original-height 6.7265in;  cropleft "0";  croptop "1";  cropright "1";
%cropbottom "0";  filename 'Galerkin.eps';file-properties "XNPEU";}} }%
%BeginExpansion
\begin{figure}[ptb]%
\centering
\includegraphics[
height=4.1667in,
width=6.0701in
]%
{Galerkin.eps}%
\end{figure}
%EndExpansion



\end{center}


\end{document}